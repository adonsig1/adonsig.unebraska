\documentclass[11pt]{article}
\usepackage{amsmath,amstext,amssymb}
% \usepackage{fancyheadings}
\usepackage{fullpage}

\def\sB{{\cal B}}
\def\sC{{\cal C}}
\def\sF{{\cal F}}
\def\sG{{\cal G}}
\def\sH{{\cal H}}
\def\sL{{\cal L}}
\def\sM{{\cal M}}
\def\sP{{\cal P}}
\def\sT{{\cal T}}
\def\bC{\mathbb{C}}
\def\bR{\mathbb{R}}
\def\bN{\mathbb{N}}
\def\bT{\mathbb{T}}
\newcommand{\dlimi}[1]{{ \lim\limits_{#1 \rightarrow \infty} }}
\newcommand{\dlim}{\lim\limits}
\newcommand{\dsum}{\sum\limits}
\newcommand{\sequ}[2][n]{{ \left( #2 \right)_{#1=1}^{\,\,\,\,\,\infty} }}


% \pagestyle{fancy}
% \lhead{AM 331/PMATH 331}
% \rhead{Final Exam, Winter '97}
\reversemarginpar
\setlength{\marginparwidth}{55pt}%
\setlength{\oddsidemargin}{20pt}%
\setlength{\evensidemargin}{20pt}%
\setlength{\textwidth}{6in}%
\begin{document}

\thispagestyle{plain}

\begin{center}
{\large \bf University of Waterloo }\\
{\large \bf Waterloo, Ontario}\\ 
\bf AM 331/PMATH 331 -- $\bR$eal Analysis \\
\large Final Exam -- Winter 1997 \\
9-12, April 23, 1997, MC 4040
\end{center}

\bigskip
\begin{verse} \bf
100 points \hfill 3 hours \\
Instructor: A. Donsig \hfill no aids \\
\end{verse}

\noindent
{\large \bf Part A}
\medskip

\begin{enumerate}

\item For each \marginpar{$18 = 6 \times 3$} of the following, either 
	carefully state the theorem or define the concept:
	\begin{enumerate}
	\item Cauchy sequence,
	\item The Extreme Value Theorem,
	\item Uniform continuity of a function $f : \bR \to \bR $,
	\item A compact subset of a normed vector space $(V,\|\cdot\|)$,
	\item The Weierstrass Approximation Theorem, and
	\item The Fourier Series of a function $f : \bR \to \bR$.
	\end{enumerate}

\item State and prove \marginpar{$12$} the Banach Contraction Princple.

\end{enumerate}

\noindent
{\large \bf Part B}
\medskip

Answer any {\bf five} of the following questions:\marginpar{$60 = 5 \times 12$}

\begin{enumerate}
\item If the sequence $\bigl( a_n \bigr)$ satisfies
	$$ a_n \le a_{n+1} \le a_n + \frac 1 {2^n} $$
	for $n=1,2,3,\ldots$, then prove that $\bigl( a_n \bigr)$
	converges and $\dlim_{n \to \infty} a_n \le a_1 + 1 $.

\item Prove that if a sequence of functions $f_n \in C[0,1]$
	converges uniformly to $f \in C[0,1]$, i.e., $\|f-f_n\|_\infty \to 0$,
	then
	$$ \lim_{n\to\infty} \int_0^1 f_n(t)\, dt = \int_0^1 f(t)\, dt.$$

\item If $A \subset V$, where $(V,\|\cdot\|)$ is a normed vector
	space, then the interior of $A$, denoted $A^0$, is the set of all
	$a \in A$ for which there is a $\delta > 0$ with
		$$ \{ v \in V : \| a - v \| < \delta \} \subset A. $$
	Prove that $A^0$ is open and if $G \subset A$ and $G$ is open, then
	$G \subset A^0$.

\item Using Chebyshev polynomials, find a polynomial of degree 6 to
	approximate $x^8-2x^7$ for $x \in [-1,1]$ and find an upper 
	bound on the error.

\item Prove that the integral equation 
	$\displaystyle f(x) = \int_0^x e^{-t} f(t) \, dt $
	has a unique solution in $(C[0,b],\|\cdot\|_\infty)$.

\item If $f(x) = x^2$, $x \in [-\pi,\pi]$, then show that
	$\displaystyle f(x) \sim \frac {\pi^2} 3 + \sum_{k=1}^\infty  
		\frac {4(-1)^k} {k^2} \cos kx $
	and use this to evaluate $\dsum_{k=1}^\infty 1/k^2$.

\end{enumerate}

\noindent
{\large \bf Part C}
\medskip

Answer any {\bf one} of the following questions: \marginpar{$10$}

\begin{enumerate}
\item Suppose that $\sequ {a_n}$ is a sequence of non-negative terms
	with $a_1 \ge a_2 \ge \cdots \ge 0$.  Prove that 
	$\dsum_{k=1}^\infty a_k$ converges if and only if 
	$\dsum_{k=1}^\infty 2^k a_{2^k}$ converges.

	\textsl{Hint:} Compare $(a_1,a_2,a_3,a_4,\ldots)$ with 
	$(a_1/2,a_2,a_4,a_4,a_8,a_8,a_8,a_8,a_{16},a_{16}\ldots)$
	and with $(a_1,a_2,a_2,a_4,a_4,a_4,a_4,a_8,a_8\ldots)$.

\item For every function $f: [0,1] \to \bR$, we have its graph in $\bR^2$,
	namely the set $G(f) = \{ (x,f(x)) : x \in [0,1] \}$.
	Show  $f$ is continuous if and only if $G(f)$ is a compact
	subset of $\bR^2$.

\item Show that for $-1 < t < 1$, we have $\displaystyle 
	\sum_{n=0}^\infty T_n(x) t^n = \frac {1-tx} {1-2tx+t^2}$.

	\textsl{Hint:} first look at the real part of $(e^{i\theta})^n$ where 
	$\cos\theta=x$.

\end{enumerate}

\bigskip
\begin{center} \bf Table of Chebyshev Polynomials \end{center}

\begin{align*}
T_0(x) &= 1\\
T_1(x) &= x \\
T_2(x) &= 2x^2   - 1 \\
T_3(x) &= 4x^3   - 3x \\
T_4(x) &= 8x^4   - 8x^2   + 1 \\
T_5(x) &= 16x^5  - 20x^3  + 5x \\
T_6(x) &= 32 x^6 - 48x^4  + 18x^2  - 1 \\
T_7(x) &= 64 x^7 - 112x^5 + 56x^3  - 7x \\
T_8(x) &= 128x^8 - 256x^6 + 160x^4 - 32x^2 + 1
\end{align*}

\end{document}

