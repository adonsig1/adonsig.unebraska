\documentclass[12pt]{newtest}
\usepackage{amsmath}
\testdate{9 June 09}
\course{Math 896-503}
\header{Assignment \# 2}
\instructions{Due: June 10th}

\newcommand{\ii}{\vec{\imath}}
\newcommand{\jj}{\vec{\jmath}}
\newcommand{\kk}{\vec{k}}
\newcommand{\vecc}[1]{\mathbf{#1}}
\newcommand{\ep}{\epsilon}
% \newcommand{\det}{\hbox{det}\,}
\newcommand{\spann}{\hbox{Span}}
\newcommand{\lip}{\left\langle}
\newcommand{\rip}{\right\rangle}
\newcommand{\rank}{\hbox{rank}\,}
\newcommand{\Nul}{\hbox{Nul}\,}
\newcommand{\Col}{\hbox{Col}\,}
\newcommand{\mesh}{\hbox{mesh}\,}
\newcommand{\bN}{\mathbb{N}}
\newcommand{\bP}{\mathbb{P}}
\newcommand{\bQ}{\mathbb{Q}}
\newcommand{\bR}{\mathbb{R}}
\newcommand{\bZ}{\mathbb{Z}}
\newcommand{\ba}{\mathbf{a}}
\newcommand{\bb}{\mathbf{b}}
\newcommand{\bx}{\mathbf{x}}
\newcommand{\by}{\mathbf{y}}
\newcommand{\sR}{\mathcal{R}}
\newcommand{\sS}{\mathcal{S}}
\newcommand{\sT}{\mathcal{T}}
\newcommand{\dsum}{\sum\limits}
\newcommand{\dlim}{\lim\limits}


\newcommand{\extra}{\textsc{Extra Credit:}\,\,}
\newcommand{\hint}{\textsc{Hint:}\,\,}
\newcommand{\bhint}{\\ \hint}


\begin{document}
\question{
Consider the function $F : \bR^2 \to \bR$ given by
\[ F(x,y) = \begin{cases} \dfrac{2x^2y}{x^4+y^2} & \hbox{if $(x,y) \ne (0,0)$,} \\
        0 & \hbox{if $(x,y) = (0,0)$.} \end{cases} \]
\begin{enumerate}
\item Show, for any straight line $L$ through $(0,0)$, the limit of
$F$ along the line $L$ is $0$.
\item Show that, for the function $\phi : \bR \to \bR^2 : t \mapsto (t,t^2)$,
$\displaystyle \lim_{t \to 0} F(\phi(t)) = 1$.
\item Is it true that $\displaystyle \lim_{(x,y)\to(0,0)} F(x,y) = 0$?
Justify your answer.
\end{enumerate}
}{
}
%\question{
%Consider the function $F : \bR^3 \to \bR$ given by $F(x,y,z)=x^2+y^2+z^2$.
%\begin{enumerate}
%\item Find the differential of $F$ at $a=(3,2,6)$, $dF_a$,
%which is a linear transformation from $\bR^3$ to $\bR^3$.
%\item Using the differential, find an approximate value for $3.02^2+1.97^2+5.98^2$.
%\end{enumerate}
%}{
%}
\end{document}
