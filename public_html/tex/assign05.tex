\documentclass[12pt]{newtest}
\usepackage{amsmath}
\testdate{16 June 09}
\course{Math 896-503}
\header{Assignment \# 5}
\instructions{Due: June 18th}

\newcommand{\ii}{\vec{\imath}}
\newcommand{\jj}{\vec{\jmath}}
\newcommand{\kk}{\vec{k}}
\newcommand{\vecc}[1]{\mathbf{#1}}
\newcommand{\ep}{\epsilon}
% \newcommand{\det}{\hbox{det}\,}
\newcommand{\spann}{\hbox{Span}}
\newcommand{\lip}{\left\langle}
\newcommand{\rip}{\right\rangle}
\newcommand{\rank}{\hbox{rank}\,}
\newcommand{\Nul}{\hbox{Nul}\,}
\newcommand{\Col}{\hbox{Col}\,}
\newcommand{\mesh}{\hbox{mesh}\,}
\newcommand{\bN}{\mathbb{N}}
\newcommand{\bP}{\mathbb{P}}
\newcommand{\bQ}{\mathbb{Q}}
\newcommand{\bR}{\mathbb{R}}
\newcommand{\bZ}{\mathbb{Z}}
\newcommand{\ba}{\mathbf{a}}
\newcommand{\bb}{\mathbf{b}}
\newcommand{\bc}{\mathbf{c}}
\newcommand{\be}{\mathbf{e}}
\newcommand{\bx}{\mathbf{x}}
\newcommand{\by}{\mathbf{y}}
\newcommand{\sR}{\mathcal{R}}
\newcommand{\sS}{\mathcal{S}}
\newcommand{\sT}{\mathcal{T}}
\newcommand{\dsum}{\sum\limits}
\newcommand{\dlim}{\lim\limits}
\newcommand{\pder}[2]{\frac{\partial #1}{\partial #2}}
\newcommand{\ppder}[2]{\frac{\partial^2 #1}{\partial #2^2}}



\newcommand{\extra}{\textsc{Extra Credit:}\,\,}
\newcommand{\hint}{\textsc{Hint:}\,\,}
\newcommand{\bhint}{\\ \hint}


\begin{document}
\question{
A unit speed parametrization of a circle may be written 
\[ c(s)=\bc+r\cos s/r \be_1 + r\sin s/r \be_2. \]
where $\be_1$ and $\be_2$ are unit length orthogonal vectors.

If $f$ is a unit-speed curve with $\kappa(0) \ne 0$, prove that 
there is one and only one circle $c$ which approximates $f$ near
$f(0)$ in the sense that $c(0)=f(0)$, $c'(0)=f'(0)$, and $c''(0)=f''(0)$.
Show that $c$ lies in the osculating plane of $f$ at $f(0)$ and 
find its center and radius.
}{
}
\question{
Suppose that $f : I \to \bR^3$ and a reparametrization $g=f\circ \alpha : J \to \bR^3$ 
are both unit-speed curves.  
\begin{enumerate}
\item Show there is $t_0 \in \bR$ such that for all $t \in J$, $\alpha(t)=\pm t+t_0$.
\item If $T_g,N_g,B_g$ is the Frenet frame field for $g$, and $\kappa_g$ and $\tau_g$
are the curvature and torsion functions for $g$, then
\[ T_g = \pm T\circ \alpha, N_g = N \circ \alpha, B_g = \pm N \circ alpha,
	\kappa_g = \kappa \circ \alpha, \tau_g = \tau \circ \alpha. \]
\end{enumerate}
}{
}
\end{document}
