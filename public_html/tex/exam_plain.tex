\magnification=\magstep2
\font\bigrom=cmr10 scaled\magstep2
\topskip=0pt
\hoffset=-0.3truein
\hsize=7truein
\nopagenumbers
\def\answer{{\vrule width1.7truein height0pt depth0.5pt}}

\noindent
\centerline{Math131-503 \hfill Exam \# 3A \hfill Spring 1992}
\vskip 0.1truein
\noindent
Name \hskip 2.6truein Id. \# 
\hrule
\vskip 1.5truein

\centerline{\bigrom Instructions: }

\item{1.}  Do not begin until told to do so.

\item{2.}  Be sure to write your name on the top of this page.

\item{3.}  SHOW YOUR WORK on the long answer questions, as 
NO WORK means NO CREDIT.

\item{4.}  There are eleven long answer questions.
The number of points each question is worth is written beside the
question.

\item{5.}  You are {\bf not} allowed to use a calculator that can do 
calculus or symbolic mathematics (for example, no HP 48's).

\item{6.}  There is a total of 100 points on the exam.  
You have fifty minutes---pace yourself.

\item{7.}  Good luck!
\vskip 1truein
\eject
\noindent
(10 pts)
Find the area bounded by the curves $ y = 2x $ and $ y = 3 -x^2 $.
\vskip 20pt

\noindent
(10 pts) 
Find the solid obtained from revolving the curve $ \displaystyle 
y = {1 \over \sqrt{x}}$ from $x=e$ to $x=e^2$ about the $x$-axis.
\vskip 20pt

\noindent
(8 pts) 
On one day in June, the temperature $t$ hours sunrise is given by
$ T(t) = 80 + 4t - 2t^2$.
Find the average temperature over the interval from sunrise until 
four hours later.
\vskip 20pt

\noindent
(8 pts) 
A flu epidemic hits College Station.
Let P(t) be the number of people sick at time $t$ where $t$ is the
time in days since the epidemic started.
If $P(0)=100$ and the flu is spreading at a rate of $20 + 5t - t^2$,
then what is the formula for $P(t)$?
\vskip 20pt

\noindent
(12 pts; each is 4) 
Evaluate each of the following:
$$\leftline{$\displaystyle {\rm a)}\qquad \int_1^4 e^{x/2}\, dx        $} $$
$$\leftline{$\displaystyle {\rm b)}\qquad \int_2^5 {5 \over x}\, dx    $} $$
$$\leftline{$\displaystyle {\rm c)}\qquad \int_{-2}^0 \sqrt{2t+5}\, dt $} $$
\vskip 20pt

\noindent
(10 pts) 
a) What value of $k$ makes the following equation true?
$$
\int 2e^{4x-1} dx = k e^{4x-1} + C
$$
b) Find a function $f$ so that $ f'(x) = \sqrt{x} + 2/x $ and $f(1)=0$.
\vskip 20pt

\noindent
(8 pts) 
Differentiate $ y = \ln(2 e^{x+5} + 5 ) $.
\vskip 40pt

\noindent
(12 pts) 
Ten kilograms of a radioactive substance with {\it decay} constant .02 is
buried underneath Sbissa.

a) Find a formula for the amount substance left after $t$ years.

b) How much is left after thirty years?

c) What is the half life of this radioactive substance?
\vskip 20pt

\noindent
(5 pts) Solve $ \ln( \ln( 3x ) ) = 0 $ for $x$.
\vskip 20pt

\noindent
(10 pts) 
The function $ f(x) = \displaystyle { {\ln(x) + 1 } \over x } $ has a relative
extreme point for $x>0$.
Find the coordinates of the point and say what kind of extreme point it is.
\vskip 20pt

\noindent
(7 pts) 
Use the midpoint rule or a Riemann sum to approximate
$$ \int_1^2 x^3 dx $$
using three equal divisions.
\vfill
\end
