\documentclass[12pt]{newtest}
\usepackage{amsmath}
\testdate{11 June 09}
\course{Math 896-503}
\header{Assignment \# 4}
\instructions{Due: June 15th}

\newcommand{\ii}{\vec{\imath}}
\newcommand{\jj}{\vec{\jmath}}
\newcommand{\kk}{\vec{k}}
\newcommand{\vecc}[1]{\mathbf{#1}}
\newcommand{\ep}{\epsilon}
% \newcommand{\det}{\hbox{det}\,}
\newcommand{\spann}{\hbox{Span}}
\newcommand{\lip}{\left\langle}
\newcommand{\rip}{\right\rangle}
\newcommand{\rank}{\hbox{rank}\,}
\newcommand{\Nul}{\hbox{Nul}\,}
\newcommand{\Col}{\hbox{Col}\,}
\newcommand{\mesh}{\hbox{mesh}\,}
\newcommand{\bN}{\mathbb{N}}
\newcommand{\bP}{\mathbb{P}}
\newcommand{\bQ}{\mathbb{Q}}
\newcommand{\bR}{\mathbb{R}}
\newcommand{\bZ}{\mathbb{Z}}
\newcommand{\ba}{\mathbf{a}}
\newcommand{\bb}{\mathbf{b}}
\newcommand{\bx}{\mathbf{x}}
\newcommand{\by}{\mathbf{y}}
\newcommand{\sR}{\mathcal{R}}
\newcommand{\sS}{\mathcal{S}}
\newcommand{\sT}{\mathcal{T}}
\newcommand{\dsum}{\sum\limits}
\newcommand{\dlim}{\lim\limits}
\newcommand{\pder}[2]{\frac{\partial #1}{\partial #2}}
\newcommand{\ppder}[2]{\frac{\partial^2 #1}{\partial #2^2}}



\newcommand{\extra}{\textsc{Extra Credit:}\,\,}
\newcommand{\hint}{\textsc{Hint:}\,\,}
\newcommand{\bhint}{\\ \hint}


\begin{document}
\question{
Define $f : \bR^2 \to \bR$ by
\[ f(x,y)=\begin{cases} \dfrac{x^2y}{x^2+y^2} & \hbox{if $(x,y)\ne (0,0)$,} \\
        0 & \hbox{if $(x,y)= (0,0)$.} \end{cases} \]
\begin{enumerate}
\item Prove that $f$ is continuous at $(0,0)$.

\hint First show that $|xy| \le x^2+y^2$ for all $x,y \in \bR$.
\item For each direction vector $v \in \bR^2$, show that
        $D_vf$ exists and compute it.
\item Show that $f$ is \textbf{not} differentiable at $(0,0)$.

\hint If it was, we could compute the directional derivatives using
        the partial derivatives at $(0,0)$.  Compare these formulae
        to your answer to part (b).
\end{enumerate}

}{
}
\question{
Do Problem 3.14 (page 89) in Edwards.  That is,
the following example illustrates the hazards of denoting functions by
real variables.  Let $w=f(x,y,z)$ and $z=g(x,y)$.  Then
\[ \pder{w}{x} = \pder{w}{x} \pder{x}{x} + \pder{w}{y} \pder{y}{x} + \pder{w}{z} \pder{z}{x}
        = \pder{w}{x} + \pder{w}{z} \pder{z}{x},
\]
since $\partial x/\partial x=1$ and $\partial y/\partial x=0$.
Hence $\partial w/\partial x\, \partial z/\partial x=0$.
But if $w=x+y+z$ and $z=x+y$, then
$\partial w/\partial z= \partial z/\partial x=1$, so we have $1=0$.
Where is the mistake?
}{
}
\end{document}
