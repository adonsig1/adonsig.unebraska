\documentclass[12pt]{newtest}
\usepackage{amsmath}
\testdate{12 Jan 09}
\course{Math 325-001}
\header{Assignment \# 0}
\instructions{Due: Jan 16th}

\newcommand{\ii}{\vec{\imath}}
\newcommand{\jj}{\vec{\jmath}}
\newcommand{\kk}{\vec{k}}
\newcommand{\vecc}[1]{\mathbf{#1}}
\newcommand{\ep}{\epsilon}
% \newcommand{\det}{\hbox{det}\,}
\newcommand{\spann}{\hbox{span}\,}
\newcommand{\lip}{\left\langle}
\newcommand{\del}{\nabla}
\newcommand{\rip}{\right\rangle}
\newcommand{\rank}{\hbox{rank}\,}
\newcommand{\Nul}{\hbox{Nul}\,}
\newcommand{\Col}{\hbox{Col}\,}
\newcommand{\mesh}{\hbox{mesh}\,}
\newcommand{\bbN}{\mathbb{N}}
\newcommand{\bbP}{\mathbb{P}}
\newcommand{\bbQ}{\mathbb{Q}}
\newcommand{\bbR}{\mathbb{R}}
\newcommand{\bbZ}{\mathbb{Z}}
\newcommand{\bzero}{\mathbf{0}}
\newcommand{\ba}{\mathbf{a}}
\newcommand{\bb}{\mathbf{b}}
\newcommand{\bc}{\mathbf{c}}
\newcommand{\bi}{\mathbf{i}}
\newcommand{\bj}{\mathbf{j}}
\newcommand{\bv}{\mathbf{v}}
\newcommand{\bx}{\mathbf{x}}
\newcommand{\by}{\mathbf{y}}
\newcommand{\bK}{\mathbf{K}}
\newcommand{\bN}{\mathbf{N}}
\newcommand{\bT}{\mathbf{T}}
\newcommand{\sR}{\mathcal{R}}
\newcommand{\sS}{\mathcal{S}}
\newcommand{\sT}{\mathcal{T}}
\newcommand{\dsum}{\sum\limits}
\newcommand{\dlim}{\lim\limits}
\newcommand{\pder}[2]{\frac{\partial #1}{\partial #2}}
\newcommand{\ppder}[2]{\frac{\partial^2 #1}{\partial #2^2}}


\newcommand{\extra}{\textsc{Extra Credit:}\,\,}
\newcommand{\hint}{\textsc{Hint:}\,\,}
\newcommand{\qhint}{\quad \hint}
\newcommand{\bhint}{\\ \hint}

\begin{document}
\question{
Do Exercise~0.1.E in the background chapter.  That is,
Which of the following statements is true?
For those that are false, write down the negation of the statement.
\begin{enumerate}
\item For every $n \in \bN$, there is an $m\in\bN$ so that $m>n$.
\item For every $m \in \bN$, there is an $n\in\bN$ so that $m>n$.
\item There is an $m\in\bN$ so that for every $n \in \bN$, $m \ge n$.
\item There is an $n\in\bN$ so that for every $m \in \bN$, $m \ge n$.
\end{enumerate}
}{
}
\question{
Do Exercise~0.2.A, (d)-(h) in the background chapter.  That is,
Which of the following statements is true?
Prove or give a counterexample.
\begin{enumerate}
\setcounter{enumii}{3}
\item $A \backslash B = B \backslash A$
\item $(A \cup B)\backslash (A \cap B) = (A \backslash B) \cup (B \backslash A)$
\item $A \cap (B \cup C) = (A \cap B) \cup (A \cap C)$
\item $A \cup (B \cap C) = (A \cup B) \cap (A \cup C)$
\item If $(A \cap C) \subset (B \cap C)$, then
$(A \cup C) \subset (B \cup C)$.
\end{enumerate}
}{
}
\question{
Do Exercise~0.2.H, in the background chapter.  That is,
Suppose that $f,g,h$ are functions from $\bbR$ into $\bbR$.
Prove or give a counterexample to each of the following statements.
\qhint Only one is true.
\begin{enumerate}
\item $f\circ g = g \circ f$
\item $f\circ(g+h) = f\circ g + f\circ h$
\item $(f+g) \circ h = f\circ h + g\circ h$
\end{enumerate}
}{
}
\question{
Do Exercise~0.2.I in the background chapter.  That is,
Suppose that $f : A \to B$ and $g : B \to A$ satisfy $g \circ f=\hbox{id}_A$.
Show that $f$ is one-to-one and $g$ is onto.
}{
}
\end{document}
