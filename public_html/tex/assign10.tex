\documentclass[12pt]{newtest}
\usepackage{amsmath}
\testdate{15 Apr 09}
\course{Math 325-001}
\header{Assignment \# 10}
\instructions{Due: April 22nd}

\newcommand{\ii}{\vec{\imath}}
\newcommand{\jj}{\vec{\jmath}}
\newcommand{\kk}{\vec{k}}
\newcommand{\vecc}[1]{\mathbf{#1}}
\newcommand{\ep}{\epsilon}
% \newcommand{\det}{\hbox{det}\,}
\newcommand{\spann}{\hbox{span}\,}
\newcommand{\lip}{\left\langle}
\newcommand{\del}{\nabla}
\newcommand{\rip}{\right\rangle}
\newcommand{\rank}{\hbox{rank}\,}
\newcommand{\Nul}{\hbox{Nul}\,}
\newcommand{\Col}{\hbox{Col}\,}
\newcommand{\mesh}{\hbox{mesh}\,}
\newcommand{\bbF}{\mathbb{F}}
\newcommand{\bbI}{\mathbb{I}}
\newcommand{\bbN}{\mathbb{N}}
\newcommand{\bbP}{\mathbb{P}}
\newcommand{\bbQ}{\mathbb{Q}}
\newcommand{\bbR}{\mathbb{R}}
\newcommand{\bbZ}{\mathbb{Z}}
\newcommand{\bzero}{\mathbf{0}}
\newcommand{\ba}{\mathbf{a}}
\newcommand{\bb}{\mathbf{b}}
\newcommand{\bc}{\mathbf{c}}
\newcommand{\bi}{\mathbf{i}}
\newcommand{\bj}{\mathbf{j}}
\newcommand{\bv}{\mathbf{v}}
\newcommand{\bx}{\mathbf{x}}
\newcommand{\by}{\mathbf{y}}
\newcommand{\bK}{\mathbf{K}}
\newcommand{\bN}{\mathbf{N}}
\newcommand{\bT}{\mathbf{T}}
\newcommand{\sR}{\mathcal{R}}
\newcommand{\sS}{\mathcal{S}}
\newcommand{\sT}{\mathcal{T}}
\newcommand{\dsum}{\sum\limits}
\newcommand{\dlim}{\lim\limits}
\newcommand{\dlimsup}{\limsup\limits}
\newcommand{\dliminf}{\liminf\limits}
\newcommand{\pder}[2]{\frac{\partial #1}{\partial #2}}
\newcommand{\ppder}[2]{\frac{\partial^2 #1}{\partial #2^2}}


\newcommand{\extra}{\textsc{Extra Credit:}\,\,}
\newcommand{\hint}{\textsc{Hint:}\,\,}
\newcommand{\qhint}{\quad \hint}
\newcommand{\bhint}{\\ \hint}
\newcommand{\note}{\textsc{Note:}\,\,}

\begin{document}
\question{
Using the $\delta-\epsilon$ condition, show that $f(x)=x^3$ is continuous
at an arbitrary $x_0 \in \bbR$.
\hint $x^3-x_0^3=(x-x_0)(x^2+xx_0+x_0^2)$.
}{
}
\question{
Define $f : \bbR \to \bbR$ by $f(x)=x \chi_{\bbQ}(x)$.  That is,
$f(x)$ is $0$ for $x$ irrational and is $x$ for $x \in \bbQ$.
Show that $f$ is continuous at $0$ and that this is the only point
where $f$ is continuous.
}{
}
\question{
\begin{enumerate}
\item Suppose $f : (a,b) \to \bbR$ is continuous.  Show that if $f(r)=0$
for each rational number $r \in (a,b)$, then $f(x)=0$ for all $x \in (a,b)$.
\item Suppose $f,g : (a,b) \to \bbR$ are continuous.  Show that if $f(r)=g(r)$
for each rational number $r \in (a,b)$, then $f=g$.
\end{enumerate}
}{
}
\question{
Suppose that $f : [0,2] \to \bbR$ is continuous and $f(0)=f(2)$.
Prove that there $x,y \in [0,2]$ so that $|x-y|=1$ and $f(x)=f(y)$.
\hint consider $g(x)=f(x+1)-f(x)$ for $x \in [0,1]$.
}{
}
\question{
Suppose that $f : [a,b] \to \bbR$ and $g: [b,c] \to \bbR$ are both continuous
and $f(b)=g(b)$.
Define $h : [a,c] \to \bbR$ by 
\[ h(x) = \begin{cases} f(x) & \hbox{if $x \in [a,b]$,}\\ g(x) & \hbox{if $x \in (b,c]$.}
	\end{cases} \]
Show that $h$ is continuous on $[a,c]$.
}{
}
\question{
\extra
Suppose that $f : \bbR \to \bbR$ satisfies $f(u+v)=f(u)+f(v)$ for all $u,v \in \bbR$.
Show that if $f$ is continuous, then there is some $m \in \bbR$ so that 
for all $x \in \bbR$, $f(x)=mx$.
\bhint start with $u$ and $v$ integers.
}{
}
\end{document}
